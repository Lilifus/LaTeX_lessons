\documentclass{article}
\title{OBHPC - maths\\CM4}
\author{William JALBY \thanks{william.jalby@uvsq.fr}}
\date{xx xxxxx 2022}
\usepackage{xcolor}
\usepackage{amssymb}
\usepackage{amsmath}
\usepackage{cancel}
\usepackage{mathtools}
\usepackage{multirow}
\begin{document}
    \maketitle
    $\begin{rcases}
        $DO $I=1,N_1\\
        \hspace*{0.6cm}$DO $J=1,N_2\\
        \hspace*{1.2cm}$DO $K=1,N_3\\
        \hspace*{1.8cm}C(I,J)=C(I,J)+A(I,K)*B(K,J)\\
        \hspace*{1.2cm}ENDDO\\
        \hspace*{0.6cm}ENDDO\\
        ENDDO\\
    \end{rcases}$
    IJK\\
    \begin{tabular}{|c|c|c|c|}
        \hline
        &INNER&IN BETWEEN&ACCES AUX TABLEAUX\\
        \hline
        IJK&DOT PRODUCT&VEC$\times$ MAT&C:Strive0 A:Ligne B:Colonne\\
        \hline
        JIK&DOT PRODUCT&MAT$\times$ MAT&C:Strive0 A:Ligne B:Colonne\\
        \hline
        IKJ&AXPY&VEC$\times$MAT&C:Ligne A:Strive0 B:Ligne\\
        \hline
        JKI&AXPY&MAT$\times$VEC&C:Colonne A:Colonne B:Strive0\\
        \hline
        KIJ&AXPY&OUTER PRODUCT&C:Ligne A:Strive0 B:Ligne\\
        \hline
        KJI&AXPY&OUTER PRODUCT&C:Colonne A:Colonne B:Strive0\\
        \hline        
    \end{tabular}\\
    $A \in \mathbb{R}^{m\times n}$\\
    $B \in \mathbb{R}^{n\times l}$\\
    $C=A,B$\\
    Produit de matrices non commutatif $A.B\neq B.A$\\
    $m\neq n$ Matrices Rectangulaires | $\in \mathbb{R}^{n\times n}$| (n'existe pas)\\
    A$\in \mathbb{R}^{n\times n}$
    B$\in \mathbb{R}^{n\times n}$\\
    Matrices carrées\\
    A.B$\neq$B.A\\
    $(A+B).C=(A.C)+(BC)$
    $A^{-1}$ n'existe pas toujours.\\
    $A\in\mathbb{R}^{n\times n}$
    $Ax=b$\\
    $x$,$b \in \mathbb{R}^{n\times 1}$\\
    A et b donnés, chercher $x\Leftrightarrow x=A^{-1}b$\\
    F.P:Commutativité OK\\
    ("+" op flottante)\\
    $a"+"b=b"+"a\\
    a"*"b=b"*"a$\\
    Associativié\\
    $(a"+"b)"+"c\neq a"+"(b"+"c)$ erreur d'arrondi\\
    Une erreur = OK\\
    Mais GROS problème d'accumulation d'erreurs.\\
    $A \ 
    \in \mathbb{R}^{n\times m}$
    $A=
    \begin{pmatrix}
        A_{11}&...&A_{1s}\\
        ...&...&...\\
        A_{r1}&...&A_{rs}
    \end{pmatrix}$
    $\begin{matrix}
        u_1+...+u_r=n\\
        v_1+...v_s=m        
    \end{matrix}$\\
    $B \ 
    \in \mathbb{R}^{m\times l}$
    $B=
    \begin{pmatrix}
        B_{11}&...&B_{1t}\\
        ...&...&...\\
        B_{s1}&...&B_{st}
    \end{pmatrix}$
    $w1+...+w_t=l$\\
    $C \ 
    \in \mathbb{R}^{n\times l}$
    $C=
    \begin{pmatrix}
        C_{11}&...&C_{1t}\\
        ...&...&...\\
        C_{r1}&...&C_{rt}
    \end{pmatrix}$\\
    DO I=1,R\\
    \hspace*{0.6cm}DO J=1,T\\
    \hspace*{1.2cm}DO K=1,S\\
    \emph{$_{BLOCK}$}$\to C_{ij}=C_{ij}+A_{ik}*B{kj}$\\
    \hspace*{1.2cm}ENDDO\\
    \hspace*{0.6cm}ENDDO\\
    ENDDO\\
    \textbf{- Matrice Dense}\\
    Une matrice dense est une matrice dans laquelle "à priori" tous les éléments sont nuls.\\
    $D\in \mathbb{R}^{n\times n}$
\end{document}