\documentclass{article}
\title{OBHPC - maths\\CM1}
\author{William JALBY \thanks{william.jalby@uvsq.fr}}
\date{20 septembre 2022}
\usepackage{xcolor}
\usepackage{amssymb}
\usepackage{amsmath}
\usepackage{cancel}
\begin{document}
    \maketitle
    \textbf{- Jumeau numerique}\\
    Un Jumeau numerique reproduit au format numerique un phenomene physique\\
    $\to$Physique : Equation : \textcolor{red}{contient erreur et approximation}\\
    $\to$Math : Equation Continu : \textcolor{red} {erreur}\\
    $\to$Info : Discretisation $\to$ Algebre Lineaire 
    $\to$ resolution $Ax=b$
    $\to$ valeur propres $Ax=\lambda x$\\
    
    \textbf{Calul differentiel}\\
    $f:\mathbb{R}\to\mathbb{R}$\\
    \underline{def 1:} $f$ est derivable au point a $\Leftrightarrow$ existe $f'(a)$ tq $lim_{h\to 0} \frac{f(a+h)-f(a)}{h}=f'(a)$\\
    \underline{def 2:}\\
    $f$ est de classe $C^0 \Leftrightarrow$ $f$ est continue. f sur $I$.\\
    $f$ est de classe $C^1 \Leftrightarrow$ $f'$ existe et est continue.\\
    $f$ est de classe $C^2 \Leftrightarrow$ $f''$ existe et est continue.\\
    $f$ est de classe $C^k \Leftrightarrow$ $f^(k)$ existe et est continue.\\
    \underline{def 3:}\\
    dérivabilité à droite:
    $lim_{h\to 0+} \frac{f(a+h)-f(a)}{h}=f_D'(a)$\\
    dérivabilité à gauche:
    $lim_{h\to 0-} \frac{f(a+h)-f(a)}{h}=f_G'(a)$\\
    Différenciabilité générale:
    $f: \mathbb{R}^n \to \mathbb{R}$\\
    $f(x_1,...,x_n)$\\
    $\exists(\alpha_1,...\alpha_n)$\\
    $lim_{||h||\to 0} \frac{f(a+h)-f(a)-\alpha_1 h_1 -...-\alpha_n h_n}{||h||}$\\
    $\vec{a}=(a_1,...,a_n)$\\
    $\vec{h}=(h_1,...,h_n)$\\\\
    \textbf{Derivé partielle par rapport à la variable i}\\
    $\vec{a}=(a_1,...,a_n)$\\
    $\vec{h}=(0,...h_i,...,0)$\\
    $\alpha_i =\frac{\partial f}{\partial_{xi}}(a)$\\
    $\lim_[h_i\to 0]\frac{f(a+h)-f(a)-\alpha_i h_i}{h_i}\to 0$\\
    $f$: Differenciable généralement au point a\\
    $\Downarrow \cancel{\Uparrow}$\\
    $f$ a des dérivées partielles au point a\\
    Gradient de $f$:\\
    (nabla)$\to\triangledown f =
    \begin{pmatrix}
        \frac{\partial f}{\partial x_1}\\
        \vdots\\
        \frac{\partial f}{\partial_{xn}}
    \end{pmatrix}$
    $\triangledown f=
    \begin{pmatrix}
        \frac{\partial f_1}{\partial x_1} &...& \frac{\partial f_p}{\partial x_1}\\
        ...&...&...\\
        \frac{\partial f_1}{\partial x_n} &...& \frac{\partial f_p}{\partial x_n}
    \end{pmatrix}$\\
    $\vec{f}: \mathbb{R}^n\to \mathbb{R}^p$\\
    $\vec{f}:(f_1,...,f_p)$
    $\lim_{||h||\to 0}\frac{||\vec{f}(\vec{a}+\vec{h}-\vec{f}(a)-\triangledown\vec{f}.\vec{h}||)}{||\vec{h}||}\to 0$\\
    Dérivée partielle de la composante j par rapport à la variable i\\
    $\frac{\partial f_i}{\partial x_i} = \beta_{ji}$
    $\lim_{hi\to 0}\frac{f_j(a+h)-f(a)-\beta_{ji}h_i}{h_i}\to 0$\\
    $f:\mathbb{R}^n\to \mathbb{R}$\\
    $\triangledown f: \mathbb{R}^n\to \mathbb{R}^n$\\
    $\begin{pmatrix}
        \frac{\partial f}{\partial x_1}\\
        ...\\
        \frac{\partial f}{\partial x_n}
    \end{pmatrix}$
    $\begin{matrix}
        \to\\
        \\
    \end{matrix}$
    $\begin{matrix}
        \frac{\partial}{\partial x_1}(\frac{\partial f}{x_1})\\
        \frac{\partial}{\partial x_2}(\frac{\partial f}{\partial x_1})\\
        \frac{\partial}{\partial x_3}(\frac{\partial f}{\partial x_2})
    \end{matrix}$\\
    \hspace*{1.5cm}
    $\begin{matrix}
        \vec{\triangledown}(\frac{\partial f}{\partial x_1}) \vdots\\
        \frac{\partial}{\partial x_n}(\frac{\partial f}{\partial x_1})
    \end{matrix}$\\
    \underline{Théoreme:} Si $f$ est de classe $C^2$\\
    $\frac{\partial}{\partial x_i}(\frac{\partial f}{\partial x_j}) =\frac{\partial}{\partial x_j}(\frac{\partial f}{\partial x_i})=\frac{\partial^2 f}{\partial x_j \partial x_j}$
    $g:\mathbb{R}^2\to \mathbb{R}$\\
    $h:\mathbb{R}^2\to \mathbb{R}$\hspace*{1cm}$\exists ? f: \mathbb{R}^2 \to \mathbb{R} tq \triangledown f=
    \begin{pmatrix}
        g\\
        h
    \end{pmatrix}$\\
    $\frac{\partial f}{\partial x_1} = g$ et
    $\frac{\partial f}{\partial x_2} = h$\\
    $\frac{\partial}{\partial x_1}(\frac{\partial f}{\partial x_2})=\frac{\partial}{\partial x_2}(\frac{\partial f}{\partial x_1})$\\
    $\frac{\partial h}{\partial x_1}=\frac{\partial g}{\partial x_2}$\\
    $\vec{f}:\mathbb{R}^n\to \mathbb{R}^p$\\
    $\triangledown \vec{f}$\\
    Jacobien$(\vec{f})=
    \begin{pmatrix}
        \frac{\partial f_1}{\partial x_1}&...&\frac{\partial f_1}{\partial x_n}\\
        ...&...&...\\
        \frac{\partial f_p}{\partial x_1}&...&\frac{\partial f_p}{\partial x_n}
    \end{pmatrix}$\\
    Pour $i\neq j$, on appelle $\frac{\partial^2 f}{\partial x_i \partial x_j}$ une dérivée croisée.\\
    $f:\mathbb{R}^n\to \mathbb{R}$\\
    Laplacien: $\Delta f= \frac{\partial^2 f}{\partial^2 x_1} +
    \frac{\partial^2 f}{\partial^2 x_2}+...+\frac{\partial^2 f}{\partial^2 x_N}$\\
    $\Delta f: \mathbb{R}^n\to \mathbb{R}$\\
    $\vec{g}: \mathbb{R}^n\to \mathbb{R}^n$\\
    Divergence: div$(\vec{g})=\frac{\partial g_1}{\partial x_1}+
    \frac{\partial g_2}{\partial x_2}+...+
    \frac{\partial g_n}{\partial x_n}$\\
    $f:\mathbb{R}\to \mathbb{R}$\\
    $f$ inconnue\\\\
    $f$ tq $G(f,f',f'',...,f^{(h)},x)=0$ Equation Differentiel Ordinaire (ODE ou EDO)\\
    $f:\mathbb{R}^n\to \mathbb{R}$\\
    $f \in C^{(k)}$\\
    $\frac{\partial{\alpha_1 +\alpha_2+...+\alpha_n}}{\partial^{\alpha_1}_{x_1}
    \partial^{\alpha_2}_{x_2} ... \partial^{\alpha_n}_{x_n}}$\\
    $(f)=\frac{\partial \vec{\alpha}}{\partial_x}(f)$\\
    $H(b,\frac{\partial \vec{\alpha}}{\partial \vec{x}}(f),\frac{\partial \vec{\beta}}{\partial \vec{x}}(f),...,\vec{x})$\\
    $H:\mathbb{R}^h\to \mathbb{R}$\\
    Equation aux dérivées partielles (PDE ou EDP)\\
    ordre: max$(d1+d2=...+dn)$\\\\\\
    \textbf{- Complément de cours:}\\
    $f:\mathbb{R}^n\to \mathbb{R}$\\
    $grad(f)=
    \begin{pmatrix}
        \frac{\partial f}{\partial x_1}\\
        \vdots\\
        \frac{\partial f}{\partial x_n}    
    \end{pmatrix}$\\
    $g:\mathbb{R}^n\to \mathbb{R}^n$\\
    $div(g)=\frac{\partial g_1}{\partial x_1}+\frac{\partial g_2}{\partial x_2}+...+\frac{\partial g_n}{\partial x_n}$\\
    $grad f:\mathbb{R}^n\to \mathbb{R}^n$\\
    $div(grad(f))=\frac{\partial}{\partial x_1}(\frac{\partial f}{\partial x_1})+\frac{\partial}{\partial x_2}(\frac{\partial f}{\partial x_2})+...+\frac{\partial}{\partial x_n}(\frac{\partial f}{\partial x_n})$\\
    $=\frac{\partial^2 f}{\partial^2 x_1}+\frac{\partial^2 f}{\partial^2 x_2}+...+\frac{\partial^2 f}{\partial^2 x_n}$\\
    $=\Delta f$\\
    divergence: \framebox{$\triangledown . g$} \\
    gradient: \framebox{$\triangledown g$}\\
\end{document}